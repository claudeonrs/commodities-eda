% Options for packages loaded elsewhere
\PassOptionsToPackage{unicode}{hyperref}
\PassOptionsToPackage{hyphens}{url}
%
\documentclass[
]{article}
\usepackage{amsmath,amssymb}
\usepackage{lmodern}
\usepackage{iftex}
\ifPDFTeX
  \usepackage[T1]{fontenc}
  \usepackage[utf8]{inputenc}
  \usepackage{textcomp} % provide euro and other symbols
\else % if luatex or xetex
  \usepackage{unicode-math}
  \defaultfontfeatures{Scale=MatchLowercase}
  \defaultfontfeatures[\rmfamily]{Ligatures=TeX,Scale=1}
\fi
% Use upquote if available, for straight quotes in verbatim environments
\IfFileExists{upquote.sty}{\usepackage{upquote}}{}
\IfFileExists{microtype.sty}{% use microtype if available
  \usepackage[]{microtype}
  \UseMicrotypeSet[protrusion]{basicmath} % disable protrusion for tt fonts
}{}
\makeatletter
\@ifundefined{KOMAClassName}{% if non-KOMA class
  \IfFileExists{parskip.sty}{%
    \usepackage{parskip}
  }{% else
    \setlength{\parindent}{0pt}
    \setlength{\parskip}{6pt plus 2pt minus 1pt}}
}{% if KOMA class
  \KOMAoptions{parskip=half}}
\makeatother
\usepackage{xcolor}
\usepackage[margin=1in]{geometry}
\usepackage{longtable,booktabs,array}
\usepackage{calc} % for calculating minipage widths
% Correct order of tables after \paragraph or \subparagraph
\usepackage{etoolbox}
\makeatletter
\patchcmd\longtable{\par}{\if@noskipsec\mbox{}\fi\par}{}{}
\makeatother
% Allow footnotes in longtable head/foot
\IfFileExists{footnotehyper.sty}{\usepackage{footnotehyper}}{\usepackage{footnote}}
\makesavenoteenv{longtable}
\usepackage{graphicx}
\makeatletter
\def\maxwidth{\ifdim\Gin@nat@width>\linewidth\linewidth\else\Gin@nat@width\fi}
\def\maxheight{\ifdim\Gin@nat@height>\textheight\textheight\else\Gin@nat@height\fi}
\makeatother
% Scale images if necessary, so that they will not overflow the page
% margins by default, and it is still possible to overwrite the defaults
% using explicit options in \includegraphics[width, height, ...]{}
\setkeys{Gin}{width=\maxwidth,height=\maxheight,keepaspectratio}
% Set default figure placement to htbp
\makeatletter
\def\fps@figure{htbp}
\makeatother
\setlength{\emergencystretch}{3em} % prevent overfull lines
\providecommand{\tightlist}{%
  \setlength{\itemsep}{0pt}\setlength{\parskip}{0pt}}
\setcounter{secnumdepth}{-\maxdimen} % remove section numbering
\ifLuaTeX
  \usepackage{selnolig}  % disable illegal ligatures
\fi
\IfFileExists{bookmark.sty}{\usepackage{bookmark}}{\usepackage{hyperref}}
\IfFileExists{xurl.sty}{\usepackage{xurl}}{} % add URL line breaks if available
\urlstyle{same} % disable monospaced font for URLs
\hypersetup{
  pdftitle={Global Commodities EDA},
  pdfauthor={Claudeon Susanto},
  hidelinks,
  pdfcreator={LaTeX via pandoc}}

\title{Global Commodities EDA}
\author{Claudeon Susanto}
\date{2023-02-01}

\begin{document}
\maketitle

\hypertarget{introduction-to-the-dataset}{%
\subsection{Introduction to the
dataset}\label{introduction-to-the-dataset}}

The data that I have chosen is \textbf{Technology Adoption} and can be
downloaded from
\href{https://github.com/rfordatascience/tidytuesday/tree/master/data/2022/2022-07-19}{this
{[}link{]}}. This data provides very useful insights and statistics on
each country's development and adoption of technology.

\begin{longtable}[]{@{}
  >{\raggedright\arraybackslash}p{(\columnwidth - 12\tabcolsep) * \real{0.1083}}
  >{\raggedright\arraybackslash}p{(\columnwidth - 12\tabcolsep) * \real{0.3750}}
  >{\raggedright\arraybackslash}p{(\columnwidth - 12\tabcolsep) * \real{0.0500}}
  >{\raggedleft\arraybackslash}p{(\columnwidth - 12\tabcolsep) * \real{0.0417}}
  >{\raggedright\arraybackslash}p{(\columnwidth - 12\tabcolsep) * \real{0.1000}}
  >{\raggedright\arraybackslash}p{(\columnwidth - 12\tabcolsep) * \real{0.2333}}
  >{\raggedleft\arraybackslash}p{(\columnwidth - 12\tabcolsep) * \real{0.0917}}@{}}
\toprule()
\begin{minipage}[b]{\linewidth}\raggedright
variable
\end{minipage} & \begin{minipage}[b]{\linewidth}\raggedright
label
\end{minipage} & \begin{minipage}[b]{\linewidth}\raggedright
iso3c
\end{minipage} & \begin{minipage}[b]{\linewidth}\raggedleft
year
\end{minipage} & \begin{minipage}[b]{\linewidth}\raggedright
group
\end{minipage} & \begin{minipage}[b]{\linewidth}\raggedright
category
\end{minipage} & \begin{minipage}[b]{\linewidth}\raggedleft
value
\end{minipage} \\
\midrule()
\endhead
BCG & \% children who received a BCG immunization & AFG & 1982 &
Consumption & Vaccines & 10.000 \\
ag\_harvester & Combine harvesters - threshers in use & AFG & 2001 &
Production & Agriculture & 2.000 \\
all\_vehicles & Total vehicles (OICA) & AFG & 2005 & Consumption &
Transport & 660000.000 \\
aluminum & Aluminum primary production, in metric tons & ALB & 1850 &
Production & Industry & 0.000 \\
atm & ATMs & ABW & 2011 & Consumption & Financial & 90.000 \\
bed\_acute & Beds for those seeking in-patient acute care & AUS & 1960 &
Non-Tech & Other & 67000.000 \\
bed\_hosp & Beds in hospitals & AFG & 1960 & Non-Tech & Hospital
(non-drug medical) & 1677.093 \\
cabletv & Households that subscribe to cable & AFG & 1992 & Consumption
& Communications & 0.000 \\
elec\_coal & Electricity from coal (TWH) & ABW & 2000 & Production &
Energy & 0.000 \\
\bottomrule()
\end{longtable}

There are 491636 observations and 7 rows in this dataset. The rows are:

\begin{enumerate}
\def\labelenumi{\arabic{enumi}.}
\tightlist
\item
  \texttt{variable}: variable name
\item
  \texttt{label}: explanation on what the variable is and what it
  measures
\item
  \texttt{iso3c}: country code
\item
  \texttt{year}
\item
  \texttt{group}: there are four groups that each variable can be
  classified as

  \begin{itemize}
  \tightlist
  \item
    Consumption: technologies that directly increase the consumer's
    utility
  \item
    Production: technologies involving making goods and services that
    consumers buy
  \item
    Creation: involves research and development process of technologies
  \item
    Non-tech: not involving technologies
  \end{itemize}
\item
  \texttt{category}: There are 9 categories of the variables, which are
  Vaccines, Agriculture, Transport, Industry, Financial, Hospital
  (non-drug medical), Communications, Energy, and Others.
\item
  \texttt{value}: value of the statistics measured
\end{enumerate}

Note that each variable has different periods in which they are
available. For example, \texttt{railpkm} has observations from year 1834
to 2019. However, \texttt{servers} only has observations from the year
2010 onwards.

Furthermore, within each year, not all countries have observations of a
variable in a given year. This means that the number of countries
observed in the data given a variable differs from year to year. For
example, in year 2020, \texttt{elec\_coal} has 68 different country
observations. However, in 1991 it only has 30 different countries.

\hypertarget{plot-1}{%
\subsection{Plot 1}\label{plot-1}}

\begin{center}\includegraphics{commodities-eda_files/figure-latex/unnamed-chunk-4-1} \end{center}

\hypertarget{insights}{%
\subsubsection{Insights}\label{insights}}

This plot depicts the surplus of steel produced by each country in 2019.
Surplus can be calculated as follows
\[\text{Surplus} = \texttt{steel_produced} - \texttt{steel_demand}\] A
negative surplus means that the country produced less than what it
needed, which most likely means the country would have to import steel
from other countries. On the other hand, a positive surplus means that
the country produced more than it demanded. The surplus could be stored
for future use or exported to other countries.

Furthermore, \texttt{internetuser} (Number of people with internet
access), which is directly correlated to the size of population, is also
plotted.

There are some observations that can be gleaned from this plot:

\begin{itemize}
\tightlist
\item
  The absolute magnitude of surplus/deficit tends to be larger in Asian
  countries (red fill) as compared to that in other continents.

  \begin{itemize}
  \tightlist
  \item
    Developed East Asian economies (China, Japan, S. Korea) are among
    countries with the largest surplus of steel production (exporters).
  \item
    Emerging South East Asian economies (Thailand, Indonesia, Vietnam,
    Philippines, Malaysia) are among countries with the largest deficit
    of steel production (importers). This is likely because large
    amounts of steel are needed for construction projects in these
    developing countries experiencing rapid economic growth, but these
    countries do not have the required productive capacity and efficient
    technologies to produce enough steel to sustain their demand.
  \end{itemize}
\item
  Similarly, the magnitude of surplus/deficit tends to be larger in
  countries where the number of people with internet access (population)
  is large, such as in China and the US. As the population becomes
  smaller, the magnitude of surplus/deficit becomes smaller too.
\item
  the US is the country with the largest deficit of steel production, so
  it is likely to be the largest steel importer.
\item
  China is the country with the largest surplus of steel production, so
  it is likely to be the biggest exporter of steel. Its surplus is twice
  than that of Japan's so it is not surprising that the US labelled
  China as currency manipulator and imposed tariffs on steel.
\item
  It is interesting to note that despite its large population, India has
  smaller magnitude of surplus compared to smaller countries. This is
  likely because it is also a developing country.
\end{itemize}

\hypertarget{design-choices}{%
\subsubsection{Design choices}\label{design-choices}}

\texttt{geom\_col} was chosen as it can possibly highlight the surplus
differences between countries better than line plot or scatter plot. I
also removed the y-axis and chose to shift the country labels closer to
the bars as doing so makes it easier for the reader to identify which
bar belongs to which country.

Regarding the colour palette, ``Set1'' was chosen from
\texttt{RColorBrewer} as the difference in hues can help highlight the
different continents better. For the \texttt{geom\_point}, I chose to
represent the variable \texttt{internetuser} using \texttt{alpha} and
\texttt{size} as doing so can highlight the contrast between population
sizes.

This plot is more suitable for a layman as it is easy to compare the
sizes of surplus using the height of \texttt{geom\_col} without any
technical knowledge.

\hypertarget{plot-2}{%
\subsection{Plot 2}\label{plot-2}}

\begin{center}\includegraphics{commodities-eda_files/figure-latex/unnamed-chunk-5-1} \end{center}

\hypertarget{insights-1}{%
\subsubsection{Insights}\label{insights-1}}

In this plot, we calculated the Herfindahl-Hirschman Index (HHI) for
each commodity in the global market to measure market concentration over
time. HHI can be defined as \[\text{HHI} = \sum_{i=1}^ns_i^2\] where
\(s_i\) is the market share percentage of firm \(i\) expressed as a
whole number, not a decimal (Source: Investopedia.com). The maximum
value of HHI is \(100^2 = 10000\) which is only achievable if the market
is a pure monopoly.

As HHI increases, market concentration is higher which means there is
less competition and can be interpreted as monopoly/oligopoly.
Similarly, lower HHI means lower market concentration which leads to
higher competition as the market is closer to the perfect competition
model. The interpretation of HHI is as follows:

\begin{itemize}
\tightlist
\item
  Highly concentrated: \(\text{HHI} \geq 2500\)
\item
  Moderately concentrated: \(1500 \leq \text{HHI} \leq 2500\)
\item
  Competitive: \(\text{HHI} \leq 1500\)
\end{itemize}

We would like to measure HHI of global commodities to determine the
competitiveness of various markets where each country acts as a
firm/supplier. We measure the market share of each country by taking the
amount produced divided by global total production. From Plot 2, we can
see that market concentrations of various global commodities have
significantly changed from over 30 years ago.

\begin{itemize}
\tightlist
\item
  For the market of electricity produced by hydrotechnology, the market
  still remained competitive in 2019, albeit there has been a slow
  increase in concentration since the early 2000's. Interestingly, there
  was a huge jump in market concentration in 2020 so the market is now
  moderately concentrated.
\item
  For the market of electricity produced by solar technology, the market
  concentration has drastically decreased from highly concentrated in
  1991 to borderline competitive in 2020.
\item
  For the market of servers, the data was only available for 2010
  onwards. It experienced a sharp drop in market concentration around
  the 2015's but market concentration rose up again to highly
  concentrated in 2020.
\item
  For the market of steel, it experienced a rapid increase in market
  concentration from 1991 to 2020 (highly concentrated).
\end{itemize}

As can be seen, market concentration changes differently over time for
different types of goods. However, it can be noted that the market
concentrations of the different goods are more similar in 2020 than in
1991 where there was a huge gap in market concentrations between Solar
Powered Electricity and Steel Production.

\hypertarget{design-choices-1}{%
\subsubsection{Design choices}\label{design-choices-1}}

Line geom is used as it would capture the trend of changing market
concentrations better than a scatterplot. Log scale is used for the
y-axis as initially the lines were concentrated on the bottom side of
the canvas. Regarding the color, ``Set1'' palette was chosen to
differentiate the lines better as other palettes are too light. I also
chose to add texts and dashed lines to label the boundaries for
different levels of market concentration ratios so that the reader can
easily interpret how concentrated a market is without looking at the
numbers. Line width and point size were also increased as otherwise they
are too thin.

This plot would be appropriate for a technical audience as otherwise it
might be difficult to understand what market concentration means and how
it can be interpreted.

\hypertarget{references}{%
\subsection{References}\label{references}}

Charles Kenny and George Yang, \& Shruti Viswanathan and Michael Pisa.
(n.d.). \emph{Technology and development: An exploration of the Data}.
Center for Global Development \textbar{} Ideas to Action. Retrieved
November 11, 2022, from
\url{https://www.cgdev.org/publication/technology-and-development-exploration-data}

Colando, S. (n.d.). \emph{7-19-2022 Tidy Tuesday: Technology
Consumption}. RPubs. Retrieved November 11, 2022, from
\url{https://rpubs.com/scolando/Tidy-Tuesday-07-19-2022}

Rfordatascience. (n.d.). \emph{Tidytuesday/data/2022/2022-07-19 at
master · rfordatascience/tidytuesday}. GitHub. Retrieved November 11,
2022, from
\url{https://github.com/rfordatascience/tidytuesday/tree/master/data/2022/2022-07-19}

\end{document}
